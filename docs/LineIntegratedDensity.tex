\documentclass{notes}

\title{A line-integrated density synthetic diagnostic}
\author{Mathias Hoppe}

\newcommand{\nhat}{\hat{\bb{n}}}

\begin{document}
	\maketitle

	In this document we describe how to simulate a line-integrated density
	measurement, given a plasma geometry and electron density $n_e(r)$
	parametrized by a flux-surface label $r$. Within the assumptions made in
	\DREAM, the electron density is a flux function. The problem of calculating
	the line-integrated density thus reduces to a geometrical problem.

	Consider a line extending through the plasma from a point $\bb{x}_0$, in
	the direction $\nhat$. The line-integrated density is given by
	\begin{equation}
		\bar{n}_e =
			\int_{\ell_1}^{\ell_2} n_e\left(r(\ell)\right)\,\dd\ell,
	\end{equation}
	where $\ell_1$ is the first point along the line-of-sight which intersects
	the plasma and $\ell_2$ is the last point.

	\section*{Determining integration limits}
	To determine $\ell_1$ and $\ell_2$, we select the 2D contour corresponding
	to the last closed flux surface and iterate through all line segments making
	up this contour. For each pair of points $(x_i,z_i)$ and $(x_{i+1},z_{i+1})$
	we must determine if there is an $\ell$ such that
	\begin{equation}
		\begin{cases}
			\min\left(x_i,x_{i+1}\right)\cos\phi \leq \xhat\cdot\left(\bb{x}_0+\ell\nhat\right) \leq \max\left(x_i,x_{i+1}\right)\cos\phi,\\
			\min\left(x_i,x_{i+1}\right)\sin\phi \leq \yhat\cdot\left(\bb{x}_0+\ell\nhat\right) \leq \max\left(x_i,x_{i+1}\right)\sin\phi,\\
			\min\left(z_i,z_{i+1}\right)\leq \zhat\cdot\left(\bb{x}_0+\ell\nhat\right) \leq \max\left(z_i,z_{i+1}\right),
		\end{cases}
	\end{equation}
	for any toroidal angle $\phi\in[0,2\pi)$. The first two conditions can
	alternatively be substituted for the more straightforward condition
	\begin{equation}
		\left[\min\left(x_i,x_{i+1}\right)\right]^2\leq
		\left[\xhat\cdot\left(\bb{x}_0+\ell\nhat\right)\right]^2 +
		\left[\yhat\cdot\left(\bb{x}_0+\ell\nhat\right)\right]^2 \leq
		\left[\max\left(x_i,x_{i+1}\right)\right]^2
	\end{equation}

	\emph{Construct a parametrization of the LCFS contour, in a 2D plane, which
		allows us to solve for when the two lines intersect?}
	
	\begin{equation}
		\begin{cases}
			\left[x_i + t\left(x_{i+1}-x_i\right)\right]^2 =
				\left(x_0+\ell n_x\right)^2 + \left(y_0+\ell n_y\right)^2,\\
			z_i + t\left(z_{i+1}-z_i\right) =
				z_0 + \ell n_z.
		\end{cases}
	\end{equation}
	Solving the second equation yields
	\begin{equation}\label{eq:t}
		t = \frac{z_0-z_i + \ell n_z}{z_{i+1}-z_i},
	\end{equation}
	which can be substituted into the first equation to yield the second-degree
	polynomial equation
	\begin{equation}
		\begin{gathered}
			\left[
				x_i + \left(z_0-z_i+\ell n_z\right)\frac{x_{i+1}-x_i}{z_{i+1}-z_i}
			\right]^2 =
			x_0^2+y_0^2 + 2\ell\left(x_0n_x + y_0n_y\right) + \ell^2\left(n_x^2+n_y^2\right)\\
			%
			\Longleftrightarrow\\
			%
			x_i^2 +
			2x_i\left(z_0-z_i+\ell n_z\right)\frac{x_{i+1}-x_i}{z_{i+1}-z_i} +
			%\left(z_0-z_i+\ell n_z\right)^2
			\left[\left(z_0-z_i\right)^2 + 2\ell n_z\left(z_0-z_i\right) + \ell^2n_z^2\right]
			\left(
				\frac{x_{i+1}-x_i}{z_{i+1}-z_i}
			\right)^2
			=\\
			=
			x_0^2+y_0^2 + 2\ell\left(x_0n_x + y_0n_y\right) + \ell^2\left(n_x^2+n_y^2\right).
		\end{gathered}
	\end{equation}
	Grouping terms by degree in $\ell$ then yields
	\begin{equation}
		\begin{gathered}
			x_i^2 - x_0^2 - y_0^2 +
			2x_i\left(z_0-z_i\right)\frac{x_{i+1}-x_i}{z_{i+1}-z_i} +
			\left(z_0-z_i\right)^2\left(\frac{x_{i+1}-x_i}{z_{i+1}-z_i}\right)^2 +\\
			%
			\ell\left[
				2x_in_z\frac{x_{i+1}-x_i}{z_{i+1}-z_i} +
				2n_z\left(z_0-z_i\right)\left(
					\frac{x_{i+1}-x_i}{z_{i+1}-z_i}
				\right)^2 -
				2\left(x_0n_x + y_0n_y\right)
			\right] +\\
			%
			\ell^2\left[
				n_z^2\left(
					\frac{x_{i+1}-x_i}{z_{i+1}-z_i}
				\right)^2 -
				n_x^2 - n_y^2
			\right] = 0.
		\end{gathered}
	\end{equation}
	This can alternatively be written as
	\begin{equation}
		a_0 + a_1\ell + a_2\ell^2 = 0,
	\end{equation}
	with
	\begin{equation}
		\begin{aligned}
			a_0 &= 
				x_i^2 - x_0^2 - y_0^2 +
				2x_i\left(z_0-z_i\right)\frac{x_{i+1}-x_i}{z_{i+1}-z_i} +
				\left(z_0-z_i\right)^2\left(\frac{x_{i+1}-x_i}{z_{i+1}-z_i}\right)^2,\\
			%
			a_1 &=
				2x_in_z\frac{x_{i+1}-x_i}{z_{i+1}-z_i} +
				2n_z\left(z_0-z_i\right)\left(
					\frac{x_{i+1}-x_i}{z_{i+1}-z_i}
				\right)^2 -
				2\left(x_0n_x + y_0n_y\right),\\
			%
			a_2 &=
				n_z^2\left(
					\frac{x_{i+1}-x_i}{z_{i+1}-z_i}
				\right)^2 -
				n_x^2 - n_y^2.
		\end{aligned}
	\end{equation}
	which has the solution
	\begin{equation}
		%    l^2 + 2*a1/(2*a2) * l + a0/a2 = 0
		% => (l + a1/(2*a2))^2 = a0/a2 + a1^2/(4*a2^2) 
		% => l = -a1/(2*a2) +/- sqrt(a1^2/(4*a2^2) - a0/a2)
		\ell_{1,2} = -\frac{a_1}{2a_2}\pm \sqrt{\frac{a_1^2}{4a_2^2} - \frac{a_0}{a_2}}.
	\end{equation}
	The different solutions correspond to intersections at different toroidal
	angles. After obtaining solutions $\ell_{1,2}$ we must substitute them back
	into equation~\eqref{eq:t} and verify that $0\leq t\leq 1$, i.e.\ that the
	solutions correspond to intersection with the finite line segment we are
	considering.

\end{document}
